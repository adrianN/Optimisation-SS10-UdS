\begin{thm} Let $A\in \{0,1,-1\}^{m\times m}$, where each column has at most one $-1$ an at most one $1$ entry. Then $A$ is totally unimodular.
\end{thm}

\begin{pr} Let $N$ be a $k\times k$ submatrix of $A$. We do induction on $k$. For $k=1$ it's trivial.

Remember, determinants can be computed like this

\[\det A = \sum_{i=1}^n a_{ij} \cdot (-1)^{i+j} \det A_{ij} = \sum_{j=1}^n a_{ij} \cdot (-1)^{i+j} \det A_{ij}\]

There are two cases

\begin{itemize}
\item If $N$ has a column with at most one non-zero entry. We can just expand on that column to reduce the determinant of $N$ to the determinant of one $k-1\times k-1$ submatrix $N'$.

\item If all columns of $N$ have 2 non-zero entries (more are impossible by definition of $A$), the sum of all rows is $0$ so $N$ is singular and the determinant is $0$
\end{itemize}
\end{pr}

A nice corollary from this theorem is the fact that the node-edge incidence matrix (rows=vertices, columns=edges, $1$ if edge is at vertex) of any digraph is TU.

\chapter{Solving Integer Programs}
\section{Cutting Planes and Cutting Plane Algorithms}
In most cases the ployhedron we describe with our integer linear program is not nicely integral.

A cutting plane algorithm tries to cut off parts of the polyhedron that are not in the integer hull of the polyhedron.

The idea here is that it's easy to find the integer hull of a halfplane. 

\[\trans a x \leq b \quad \Rightarrow \quad \trans a x \leq \lfloor b \rfloor\]

where the inequality is scaled s.t. $a$ is integral and $\gcd(a)=1$. Rounding down the right hand side shifts the hyperplane such that it intersects with the closest feasible integer point.

We could to this transformation for every valid constraint for the polyhedron. However there are quite a few and we'd need to iterate the process because the new inequalities change the set of valid constraints.

Instead we just try to cut off things around the optimal solution until it becomes integral. That means we don't have to care about the other parts of the polyhedron and hopefully speeds up the process.

To find the right cutting planes we first solve the linear program relaxation to get some $x^*$. The we find some hyperplane that cuts off $x^*$ without touching the integer hull, i.e. it is valid for $P_I$, and add it to our LP.

\begin{lstlisting}
repeat
	$x^*$ = Solve LP $\max \{cx|x\in P\}$
	if $x^*$ is integral return $x^*$
	$H$ = halfspace that separates $x^*$ from $P_I$
	$P=P\cap H$
\end{lstlisting}

Of course we don't know how to find the cutting plane that cuts off $x^*$, or even how to check if a given inequality is valid for $P_I$. It's also not clear if this algorithm always terminates and if it terminates how long it takes.

\subsection{Cutting plane proofs}

There are many different ways to find cutting planes. In the following we will deal with Gomory-Chvatal cuts.

We want to prove that some inequality $\trans cx\leq \delta$ is valid for all integral solutions of $Ax\leq b$

\begin{Ex}
Let the system be

\[\begin{pmatrix}
2 & 3\\
2 & -2 \\
-6 & -2\\
-2 & -6\\
-6 & 8
\end{pmatrix}\cdot x \leq \begin{pmatrix} 27\\7\\-9\\-11\\21\end{pmatrix}\]

We look at linear combinations of the rows. For example we see by scaling rows 5:

\[-3x_1+4x_2 \leq \frac{21}{2}\]

This means we can round down $21/2$ to $10$ because $x_1,x_2$ are integral. From adding two times that new inequality to three times the first one we get

\[17x_2 \leq 101 \quad \Rightarrow \quad x_2 \leq \left\lfloor \frac{101}{17}\right\rfloor = 5\]
\end{Ex}

In general we want to find inequalities of the form

\[\trans y A x \leq \left\lfloor \trans y b \right \rfloor,\quad y\geq 0, \trans y A \text{ integral}\]

These inequalities are called Gomory-Chvatal cutting planes.

\begin{Def} Let $Ax\leq b$ system of $m$ linear inequations and $\trans cx\leq \delta$ an inequality. A sequence of linear inequalities $c_1x\leq \delta_1, c_2x\leq \delta_2,\ldots, c_mx\leq \delta_m$ is called a \emph{cuting plane proof} of $\trans c x\leq \delta$ from $Ax\leq b$ if
\begin{itemize}
\item $c_1,\ldots c_m$ are integral
\item $c_m=c$ and $\delta_m=\delta$
\item $\forall i\in [1,m]: c_ix\leq \delta_i'$ is a nonnegative linear combination of inequalities from the system

\[Ax\leq b, c_1x\leq \delta_1,\ldots,c_{i-1}x\leq \delta_{i-1}\]

and $\delta_i=\lfloor \delta_i'\rfloor$.
\end{itemize}
\end{Def}

\begin{thm}[Existence] Let $P=\{x\in \R^n | Ax\leq b\}$ be bounded and non-empty

\begin{enumerate}
\item if $P_I\neq \emptyset$ and some inequality $\trans c x\leq \delta$ with integral $c$ is valid, then there is a cutting plane proof
\item if $P_I=\emptyset$ then there is a cutting plane proof of 
\[0x\leq -1\]
\end{enumerate}
\end{thm}

\begin{thm}[Finite Length] Let $Ax\leq s$ be a rational system of linear inequalities with at least one integral solution If $\trans cx\leq \delta$ is valid for integral $c$, then $\trans cx\leq \delta$ has a cutting plane proof of finite length.
\end{thm}

While looking for Gomory-Chvatal cuts we usually use the cuts we already added to form new cuts. If we don't do that but instead add all cuts that we can form without taking into account the new constraints we form the Chvatal closure of the polyhedron.

\begin{Def} Let $P=\{x|Ax\leq b\}$ be a rational polyhedron. Adding all Gomory-Chvatal cuts $\trans yAx\leq \lfloor \trans y b\rfloor$ for $y\geq 0$ and $\trans y A$ integral, yields the \emph{Chvatal Closure}
\end{Def}

\begin{thm} The Chvatal Closure is a polyhedron\end{thm}