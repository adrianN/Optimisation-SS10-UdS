\section{The Cutting Stock Problem}
We're given $r$ rods of length $L$ and have a demand of $d_i$ pieces of length $l_i$. We want to cut the rods s.t. we meet the demand with a minimal number of raw rods.

Clearly

\[K:=\sum \frac{d_il_i}{L}\]

is a lower bound for the number of rods we need.

\subsection{Greedy Heuristic}

We will start with a greedy approach that guarantees a $2k+1$ approximation.

Initially all rods have the full length. In order to cut a piece we use an already used rod whenever possible, otherwise we start a new rod.

The analysis goes as follows: Let $G$ be the number of rods used by the algorithm. For each $i$ between $1$ and $G$ let $u_i$ be the used part and $w_i$ the waste. W.l.o.g. ordered decreasingly: $w_1\geq w_2\geq \ldots$. When we started to use the $j$th rod there wasn't enough room to use the $j-1$th. Thus $u_j\geq w_{j-1}$. We can write

\begin{align*}
G\cdot L &=  \sum_{j=1}^G (u_j+w_j)\\
	&= \sum_{j=1}^G u_j + \sum_{j=2}^G u_j +w_G\\
	&\leq L+2\sum_{j=1}^G u_j\\
	&\leq L + 2KL
\end{align*}

Thus $G\leq 2K+1$.

An alternative analysis distinguishes between "long" pieces that are longer than $L/2$ and "small" pieces. Assigning every long piece to a single rod assures that those are at least 50\% filled. We can aggregate small pieces on one rod until it is at least 50\% full, because small pieces are smaller than $L/2$. So in both cases we have a $2K+1$ approximation.

\subsection{Using an ILP}

We introduce variables $z_1,z_2,\ldots , z_t$, $t = \sum d_i$ s.t. 

\[z_j=\begin{cases} 1 & \text{rod j is used}\\
0 & \text{otherwise}\end{cases}\]

we also need variables $x_{ij}$ that indicate that we can fit $x_{ij}$ many pieces of length $l_i$ into the $j$th rod (if it is used).

Thus we get the ILP

\begin{align*}
\min \quad & \sum z_j\\
\text{s.t.} \quad & \sum x_{ij}l_i \leq Lz_j\\
	& \sum_{j} l_ix_{ij} \geq d_i && \forall i
	&x_{ij} \in \N, z_j \in \{0,1\}
\end{align*}

This isn't a particularly good formulation because it contains a lot of symmetry: there are many possibilities to use $k$ rods if we have to specify \emph{which} $k$ to use. Problems with many optimal solutions are hard to solve, in general.

Obviously the LP relaxation of this program has the trivial optimal value $K=\sum \frac{d_il_i}{L}$, as we are allowed to cut fractionally and can thus fill our rods optimally. The following assignment to the LP variables does just that:

\[z_j = \begin{cases}
1 & 1 \leq j \leq \lfloor K\rfloor \\
K-\lfloor K \rfloor & j=\lfloor K\rfloor +1\\
0 & \lfloor K \rfloor +1\end{cases}\qquad x_{ij} = \frac{d_i}{K}z_j\]

\subsection{Cutting Patterns}

But we can get a better approximation by a more sophisticated ILP.

\begin{Def}[Cutting pattern] A cutting pattern is a set of numbers $(a_1,\ldots,a_k)$ s.t. $a_i\in \N$ and $\sum a_il_i\leq L$ and the sum is maximal, i.e. there is no more room left for any $l_i$.\footnote{The book by Bertsimas doesn't make this restriction}
\end{Def}

Let $P$ be the set of all cutting patterns (this is a very large set). We introduce a variable $x_p$ for every $p\in P$ that indicates the number of patterns $p$ that we used.

This gives us the following ILP

\begin{align*}
\min \quad & \sum x_p\\
s.t. \quad & \sum a_{p_i}x_{p_i} \geq d_i && \forall i
\end{align*}

\begin{Ex} $L=7$, $l_1=2$ $l_2=3$

\[P=\{(3,0),(2,1),(0,2)\}\]

This gives us the ILP

\begin{align*}
\min \quad & x_1+x_2+x_3\\
s.t.\quad & 3x_1+2x_2+0x_3 \geq d_1\\
	&0x_1 + 1x_2+ 2x_3 \geq d_2
	&x\in \N
\end{align*}
\end{Ex}

There are different rounding strategies that produce integer solutions from the LP

\begin{itemize}
\item round up
\item round down and use the greedy heuristic to satisfy the demands
\item add cuts, e.g. Gomory cuts
\item branch and bound
\end{itemize}

But how do we solve the LP anyway? $P$ may be huge.

The idea is to start with a small number of patterns that guaranteees the existence of a solution, e.g. one cut per $l_i$. Let $P'\subset P$ be a set of patterns that is currently in use. We know the optimal LP solution for $P'$ and would like to know if it optimal for $P$. We can view the $P\backslash P'$ cuts to have non basic variables, i.e. they are 0. Now we can just use the normal stopping criterion of the simplex algorithm and check whether the reduced cost is $\geq 0$.

The reduced cost can be computed as usual. Let $y_i$ be the multiplier of the $i$th row, i.e. the value of the corresponding dual variable. Then the reduced cost is the column $p$ is eual to

\[1-\sum_{i=1}^k y_ia_i\]

($1$ because the cost for all patterns is $1$) and there is a variable of negative reduced cost iff there is a vector $a_1,\ldots,a_k$ such that the constraint $\trans a l_i \leq L$ and $\sum y_ia_i \geq 1$

We can find that vector by solving a knapsack problem. We want to grab some items with weight $a$, s.t. we don't exceed a bound $L$ and want to maximise the profit. 

Although knapsack is NP-complete it is usually simple to solve in practice. One can use a dynamic programming based approach.

\paragraph{Knapsack via Dynamic Programming} We have a table with columns from $0$ to $W$ and rows $1$ to $k$ where $W$ is the capacity and $k$ is the number of items. In cell $(i,j)$ we write the maximal value of a knapsack with capacity $j$ using only items $0,\ldots, i$. The first row is easy. Suppose we have solved the first couple of rows. Then we can compute a new cell by deciding whether we want to use item $i$ or not. 

\begin{align*}
A(1,z) &= \begin{cases} c_1 & z\geq w_1\\
0 & \text{otherwise}\end{cases}\\
A(i,z) &= \max (A(i-1,z), A(i-1,z-w_i)+c_i)
\end{align*}

The second term in the maximisation only exists if $z\geq w_i$.

If $W$ is reasonably small, filling the table is fast.