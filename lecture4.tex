\subsection{The simplex algorithm}
Recall that an LP is in standard form if

\begin{align*}
\min cx\\
Ax = b\\
x\geq 0
\end{align*}

and $A\in \R^{m\times n} has $m$ linearly independet rows. A basis $B\subseteq [1,n]$ is as set of $m$ indices such that $A_B$ has full rank $m$. It induces a basic solution

\[\begin{cases} x_B = A^{-1}_Bb\\ x_{\bar B} = 0\end{cases}\]

We say it is feasible if $x_B\geq 0$. Note that the $x_{\bar B}=0$ part makes $n-m$ non-negativity constraints active so that we actually are at a corner with $n$ active constraints in $n$ dimensions.

The simplex algorithm works by moving from corner to corner, always in the direction of better costs. At first we'll have some simplifying assumptions, to avoid the messy details:

\begin{enumerate}
\item The LP is in standard form (for conversion see \ref{Sec:standardForm})
\item Every feasible basis $A$ is non-degenerate. A basis $B$ is non-degenerate if $\forall b:\ x_B=A^{-1}_Bb \gneq 0$. Note the difference to $x_B\geq 0$ for general basic solutions. So it is forbidden that more than $n$ (those in $B$ and $>n-m$ non-negativity) constraints are active (in $n$ dimensions they can't all be linearly independet of course). 
\item We are given an initial feasible basis
\item The feasible region of the LP is bounded
\end{enumerate}

In the next lectures we'll remove those assumptions one after the other.

The algorithm works as follows:
\begin{lstlisting}
SIMPLEX-TAKE-I(A,b,c)

B <- find some feasible basis
// B=$\{b_1,b_2,\ldots, b_m\} \subseteq [1,\ldots, n]$
repeat 
    for $j\in [1,n]\backslash B$ and  $b_i \in B$
        D = B $\cup$ j $\backslash$ $b_i$
        if D is a basis then
            $x_B$ = $A^{-1}_B$b
            $y_D$ = $A_D^{-1}b$ // a bfs
            if ($y_D$ $\geq$ 0 and $c_Dy_D < c_Bx_B$)
                B = D
until B hasn't changed                
\end{lstlisting}

$D$ is build by removing one constraint from $B$ and adding another one that wasn't in there before. We then check if $D$ is non-singular. If so it induces a basic solution, we just have to check if it's feasible. If yes we check if the cost improves and move to $D$ if it does.

It will later turn out that we don't actually have to check every combination for $j$ and $b_i$. The choice of $b_i$ is uniquely determined.

We can't get the same basic solution from two different bases, if we have a non-degenerate LP. In the degenerate case this may very well happen:

\begin{align*}
x_2 + x_3 &= 1\\
x_1 - x_2 +x_4 &=0\\
x_1 + x_2 +x_5 &= 2\\
\end{align*}

% figure

The above system gives us the solution $(1,1,0,0,0)$ for three bases $\{1,2,3\},\{1,2,4\},\{1,2,5\}$. This happens because we have a point where three constraints are active, although we're in 2D. See figure \ref{Fig:degenerateLP}. It is easy to see that if we have two bases build from indices $k\in [1,\ldots,n]$ that give us the same solution one of the $x_k$ (a basic variable) has to be zero. %easy to see :S

%figure

We move from solution $x$ to solution $y$ along some vector $d$ ($y=x+\Theta d$). Have a look at figure \ref{Fig:movingToSolutions}. We do some observations on the vector $d$. 

\begin{itemize}
\item $d_k=0$ if $k\not \in B \cup j$, because we assumed all variables not in the basis (i.e. the non-basic variables) to be zero to make the non-negativity constraints active 
\item $d_j>0$ $(x_k=0, y_k=0)$. By choosing $\Theta$ accordingly we can scale this to be $d_j=1$
\item $Ad = 0$ ($Ad = $(Ay-Ax)/\Theta = (b-b)/\Theta = 0)
\end{itemize}

By using these observations we can directly calculate the $b_i$ we need to remove.

We can rewrite $A_d$ like follows and can use that to find the components of $d$:

\[A_d = A_B d_B+A_j = 0 \Rightarrow d_B = -A^{-1}_B A_j\]

We also want $y\geq 0$ so that we need

\[x_B - d_B = x_B - \Theta A_B^{-1} A_j \geq 0\]

Thet means if $d_{b_i}>0$ we are ok, however if $d_{b_i} <0$ then $x_{b_i} +\Theta_{b_i} \stackrel{!}{\geq} 0$ so $\Theta \leq \frac{-x_{b_i}}{d_{b_i}}. We can do that by choosing

\[\Theta = \min_{{b_i\in B}\atop {d_{b_i} <0}} \left| \frac{x_{b_i}}{d_{b_i}}\right |\]

So the indices attaining the minimum will be the ones we'll have to take out. Because we assumed that the system is non degenerate there will be only one element actually attaining it. %why does non-degenerate imply that?

How does the cost change?

\[cy -cx = \Theta cd = \Theta(c_j+c_Bd_B) = \Theta(c_j - c_B^TA^{-1}_B A_j)\]

If $y$ was a solution that we move to we did that because $cy$ was smaller that $cx$. So $c_j - c_B^TA^{-1}_B A_j$ should be negative if we want to move to $y$

\begin{lstlisting}
SIMPLEX-TAKE-II (A,b,c)

B = some feasible basis
repeat 
    for $j\in \{1,\ldots,n\}\backslash B$ // $b_i$ is determined
        $d_B$ = $A^{-1}_B A_j$
        if $c_j - c_B^TA^{-1}_B A_j < 0$
            $b_i$ = index  s.t. $d_{b_i} <0$, 
                    minimizing $\left| \frac{x_{b_i}}{d_{b_i}}\right|$
            B = B $\cup$ j $\backslash$ $b_i$
until B hasn't changed
\end{lstlisting}

%todo rechnungen zwischen II und III

\begin{lstlisting}
SIMPLEX-TAKE-III (A,b,c)

B = some feasible solution
repeat
    $\bar c$ = c - $c^T_B A^{-1}_BA$ //reduced cost vector
    if $\exists j:{\bar c}_j<0$ 
        u = $A^{-1}_B A_j$ // dimension m
        $b_i$ = index in B s.t. $u_i >0$ 
                minimizes  $x_{b_i}/u_i$
        B = B $\cup$ j $\backslash b_i$ 
until B hasn't changed 
// $\bar c \geq 0$ 
\end{lstlisting}

The algorithm always terminates because the assumptions ensure that we always have a optimal solution and we improve our solution in every step.

\begin{thm}\label{Pr:simplexIIIopt} Let B be a basis. If $x_B=A^{-1}_Bb\geq 0$ and $\bar c=c-c_B^{T}A_B^{-1}A_j$ then $B$ is optimal.\end{thm}

\begin{pr}[Theorem \ref{Pr:simplexIIIopt}] Let $x$ be the bsf induced by $B$, $y$ be some feasible solution. We want to argue that the cost of $x$ is smaller than the cost of $y$. Let $d=y-x$ then 
\[cd = c_Bd_B + \sum_{j\in \bar B} c_jd_j \] 

To be continued
\end{pr}